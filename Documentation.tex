% Options for packages loaded elsewhere
\PassOptionsToPackage{unicode}{hyperref}
\PassOptionsToPackage{hyphens}{url}
%
\documentclass[
]{article}
\usepackage{lmodern}
\usepackage{amssymb,amsmath}
\usepackage{ifxetex,ifluatex}
\ifnum 0\ifxetex 1\fi\ifluatex 1\fi=0 % if pdftex
  \usepackage[T1]{fontenc}
  \usepackage[utf8]{inputenc}
  \usepackage{textcomp} % provide euro and other symbols
\else % if luatex or xetex
  \usepackage{unicode-math}
  \defaultfontfeatures{Scale=MatchLowercase}
  \defaultfontfeatures[\rmfamily]{Ligatures=TeX,Scale=1}
\fi
% Use upquote if available, for straight quotes in verbatim environments
\IfFileExists{upquote.sty}{\usepackage{upquote}}{}
\IfFileExists{microtype.sty}{% use microtype if available
  \usepackage[]{microtype}
  \UseMicrotypeSet[protrusion]{basicmath} % disable protrusion for tt fonts
}{}
\makeatletter
\@ifundefined{KOMAClassName}{% if non-KOMA class
  \IfFileExists{parskip.sty}{%
    \usepackage{parskip}
  }{% else
    \setlength{\parindent}{0pt}
    \setlength{\parskip}{6pt plus 2pt minus 1pt}}
}{% if KOMA class
  \KOMAoptions{parskip=half}}
\makeatother
\usepackage{xcolor}
\IfFileExists{xurl.sty}{\usepackage{xurl}}{} % add URL line breaks if available
\IfFileExists{bookmark.sty}{\usepackage{bookmark}}{\usepackage{hyperref}}
\hypersetup{
  pdftitle={Documentation},
  pdfauthor={Lasse Kehrhahn},
  hidelinks,
  pdfcreator={LaTeX via pandoc}}
\urlstyle{same} % disable monospaced font for URLs
\usepackage[margin=1in]{geometry}
\usepackage{graphicx,grffile}
\makeatletter
\def\maxwidth{\ifdim\Gin@nat@width>\linewidth\linewidth\else\Gin@nat@width\fi}
\def\maxheight{\ifdim\Gin@nat@height>\textheight\textheight\else\Gin@nat@height\fi}
\makeatother
% Scale images if necessary, so that they will not overflow the page
% margins by default, and it is still possible to overwrite the defaults
% using explicit options in \includegraphics[width, height, ...]{}
\setkeys{Gin}{width=\maxwidth,height=\maxheight,keepaspectratio}
% Set default figure placement to htbp
\makeatletter
\def\fps@figure{htbp}
\makeatother
\setlength{\emergencystretch}{3em} % prevent overfull lines
\providecommand{\tightlist}{%
  \setlength{\itemsep}{0pt}\setlength{\parskip}{0pt}}
\setcounter{secnumdepth}{-\maxdimen} % remove section numbering

\title{Documentation}
\author{Lasse Kehrhahn}
\date{}

\begin{document}
\maketitle

Funktionen sind \emph{kursiv}.\\
Reiter sind \textbf{fett}.

\hypertarget{init}{%
\section{01 INIT}\label{init}}

Input-Parameter eingeben.

Kunden: 2\\
Kundenwünsche: 2\\
Functional Requ.: 3\\
Processes: 3\\
Ressourcen: 4

\hypertarget{gen_ead}{%
\section{gen\_EAD}\label{gen_ead}}

Die eingegebenen Parameter werden aus \textbf{01 INIT} übergeben.

\hypertarget{demand-generation}{%
\subsubsection{Demand generation}\label{demand-generation}}

Der Demand ist der Startwert, also die ``Messgröße'' für die Kunden C.\\
Der Demand wird in Form eines Vektors zufällig generiert, hierzu wird
die Funktion \emph{gen\_Demand} aus \textbf{gen\_Q.R} ausgeführt.

Daraus ergibt sich: 70, 31. (In Summe 101, wieso?)

\hypertarget{create-designmatrix}{%
\subsubsection{Create Designmatrix}\label{create-designmatrix}}

Die Designmatrizen werden zufällig generiert. Hierzu wird die Funktion
\emph{create\_designmatrix} aus dem Reiter \textbf{designfunctions.R}
ausgeführt.

\hypertarget{customer-diversity}{%
\subsubsection{Customer diversity}\label{customer-diversity}}

Daraus ergibt sich: \[
demand*A_{CCN} = CN.
\]

\[
\left(\begin{array}{cc} 
70\\
31
\end{array}\right)
* \left(\begin{array}{cc} 
0 & 1\\
1 & 0
\end{array}\right)
= \left(\begin{array}{cc} 
31\\ 
70
\end{array}\right)
\]

\end{document}
